\chapter{Introducci\'on}


\section{Antecedentes � Realidad Problem\'atica}


\section{Definici�n del Problema}



\section{Objetivos}

\subsection{Objetivos Espec�ficos}



\section{Justificaci�n}


\section{Alcances}

El alcance se refiere a las grandes actividades que se va desarrollar en la tesis. Estas son por lo general:
- Estudio del arte del problema (revisi�n de todo lo que existe sobre el problema de la tesis).
- An�lisis, desarrollo y puesta en marcha de una soluci�n tecnol�gica (software).
- Prueba del sistema.
Es deseable que la propuesta sea verificada en una organizaci�n, el cual constituye el estudio de caso. En este sentido se deber� indicar la organizaci�n (empresa, instituci�n, industria, sector de gobierno, etc.) que ser� beneficiada.
