\chapter{Estado del Arte}\label{chap:estadodelarte}

Objetivo: El informe del estado del arte permite evaluar el nivel de conocimiento del candidato respecto del problema que desea resolver. El estado del arte describe todo aquello que otros investigadores han estudiado acerca del problema. La redacci�n de la misma corresponde a uno o dos cap�tulos del informe final de su tesis

Consideraciones: El estado del arte est� conformado por la definici�n del problema, las variaciones del problema (si existiera), la taxonom�a (en donde se encuentra el problema de estudio), las metodolog�as, modelos, algoritmos, aplicaciones, aplicativos (software), plataformas asociados directamente al problema, esto es, aplicado al problema de estudio.
No se debe confundir el estado del arte con el marco te�rico. El estado del arte se centra al  problema de estudio. Entretanto el marco te�rico es dado por los fundamentos alrededor del tema de tesis

Estructura del informe: La estructura del estado del arte debe incluir en lo posible:
�	Taxonom�a (variantes o clasificaci�n del problema)
�	M�todos / Modelos / Procedimientos / Buenas Pr�cticas
�	Algoritmos
�	Aplicativos/ Arquitecturas/ Software
�	Casos de estudios/ aplicaciones
�	Marco legal/ Normas/ / Regulaciones
Taxonom�a: La taxonom�a consiste en la clasificaci�n del problema, y tiene por objetivo que el lector tenga un panorama mayor sobre el problema en estudio. Algunas veces no es posible hacer una taxonom�a.
Aplicaciones: Describir las diversas aplicaciones (uso) existentes en la literatura. Por ejemplo, el tema �algoritmos gen�ticos para la optimizaci�n de cortes en una dimensi�n�, tiene diversas aplicaciones como: cortes de fierros, cortes de aluminio, cortes de bobina de papel, etc.
M�todos / Modelos / Buenas Pr�cticas: Describir sucintamente Deber� ser presentado los diversos m�todos/ modelos/ Buenas pr�cticas que han sido desarrollado para resolver el problema en estudio.
Aplicativos (software): Describir los diversos software que existen en la literatura y en el �mbito comercial. Se�alar las bondades, restricciones, y el m�todo en la implementaci�n del software (si lo hubiera).
